\chapter{Discussion}

In the discussion chapter, I highlight the main finding from results chapter and and then specifically address the three research questions posed fo rthe current dissertation. 

%The two primary purposes of this dissertation were to examine the performance of the three methods in the context of learning progressions to examine whether the use of more complicated models provide more meaningful inferences of the student learning than the use of raw data and to examine the data across three models for validation of learning progressions. 

Although learning progressions have attracted attention as a promising tool to provide feedback on student learning and to inform decisions on standards construction and curriculum development, the modeling attempts to validate developed LPs and to provide probabilistic feedback on student learning have stayed relatively small. Both IRT and diagnostic classification frameworks provide useful tools for modeling LPs by linking the theory embodied in a progression, tasks that provide evidence about a student’s level on that progression, and psychometric models that can characterize the relationship between student performance and levels on learning progressions. There are a number of psychometric models that are capable of fulfilling these modeling expectations but current applications of these psychometric models are mostly based on large scale or simulated data. Learning progressions assessments developed to account for individual student differences in cognitive strategies or skill profiles provide data which can yield detailed information on student understanding with fitting models. Thus, it would be prudent for researchers to examine viability of the models in the context of learning progressions. 

For this purpose, I examined the viability of three models; Partial Credit Model \cite{Masters1982,EmbretsonReise2000} from IRT framework and Attribute Hierarchy Model (as modified by \cite{AlonzoSteedle2009) and General Diagnostic Model (\cite{vonDavier2008}) from latent class framework. 

